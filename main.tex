\documentclass[a4paper,11.5pt,]{article}
\usepackage[utf8]{inputenc}
\usepackage[T1]{fontenc,url}
\usepackage[english,]{babel}
\usepackage{blindtext}
\usepackage{natbib}
\usepackage{gensymb}
\usepackage{amsmath}
\usepackage{changepage}
\usepackage{amssymb}
\usepackage{commath}
\usepackage{physics}
\usepackage{multicol}
\usepackage{float}
\usepackage{listings}
\usepackage{graphicx}
\usepackage{hyperref}
\usepackage{svg}
\usepackage{wrapfig}

\usepackage{multicol}
%\setlength{\columnsep}{1cm}

\usepackage{geometry}
 \geometry{
 a4paper,
 total={170mm,257mm},
 textheight =  592mm,
 left=20mm,
 right=20mm,
 tmargin=20mm,
 bmargin=20mm
 }
 \setlength{\columnsep}{20pt}
          

 \title{Stellar Spectra A. Basic Line Formation\\
 AST4310 - Radiative Processes in Astrophysics}
 \date{\normalsize{14. September 2018} }
 \author{\textsc{\small{Metin San}}}
 
 
\begin{document}

\maketitle
\begin{center}
\textsc{Introduction}
\end{center}


\begin{adjustwidth}{1cm}{1cm}

Spectral lines in stellar spectra make out the foundation of astrophysics. They provide a wealth of knowledge about the studied object and are therefore one of the most valuable sources of information to an astrophysicist. In order to get an understanding of spectral lines we will follow the work of Annie Cannon, Cecilia Payne and Marcel Minnaert which are some of the early pioneers of spectral astronomy. The topics we will consider are spectral classification, where we retrace the work of Annie Cannon, Saha-Boltzmann modeling with Cecilia Payne, and finally, modeling the strength of Fraunhofer lines with Marcel Minnaert.
\end{adjustwidth}
\begin{multicols}{2}

\begin{center}
\textsc{1. Somethingsomething}
\end{center}




\end{multicols}
\end{document}